\documentclass[a4paper]{article}

\def\doctitle{实验二十七\ 交流电桥}

\def\docabstract{交流电桥是测量各种交流阻抗的基本仪器,如电容的电容量,电感的电感量等。此外还可利用交流电桥平衡条件与频率的相关性来测量与电容、电感有关的其他物理量,如互感、磁性材料的磁导率、电容的介质损耗、介电常数和电源频率等,其测量准确度和灵敏度都很高,在电磁测量中应用极为广泛。这次实验将测量电容、电感的电容值及电容量,以及磁环的一些物理量,还有标准互感器的互感值。}

\def\dockeywords{交流电桥}

\usepackage{see-exp}

\def\doctoc{false}

\begin{document}

%title
\input{/home/supercgor/texmf/exp-titlepage.tex}

%contents here
\section{实验概要}
\subsection{实验目的}
\begin{enumerate}
    \item 学会使用交流电桥测量电容和电感及其损耗;
    \item 了解交流桥路的特点和调节平衡的方法。
\end{enumerate}

%
\subsection{实验仪器}
\begin{enumerate}
    \item 函数信号发生器;
    \item 电阻箱3个、十进制电容箱、十进制电感箱;
    \item 待测电容(纸电容、电解电容)、待测电感、待测磁环;
    \item 数字万用表。
\end{enumerate}


\subsection{实验原理}
\hspace{2em}交流电桥与直流电桥结构类似,但平衡条件不一样。在交流电桥中,平衡还需要考虑相位的影响,条件如下:\par
\begin{align*}
    \frac{|Z_1|}{|Z_2|} & =\frac{|Z_3|}{|Z_4|} \\
    \varphi_1-\varphi_2 & =\varphi_3-\varphi_4
\end{align*}
\hspace{2em}  \par

其中$z_1,z_2,z_3,z_4$是电桥元件的阻抗,$\varphi_1、\varphi_2、\varphi_3、\varphi_4$是阻抗的相位,电桥标示如下图所示:

\section{实验数据及分析}

%%%%%%%%%%%%%%%%%%%%%%%%%%%%%%%%%%%%%%%%%%%%%%%%%%%%%%%%%%%%%%%%%%%

\end{document}